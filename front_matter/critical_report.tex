\documentclass[shorttitlesize=55]{ees}

\begin{document}

\eesTitlePage

\eesCriticalReport{
    & –    & –      & This mass by Reinhardt has been substantially revised by
                      Jan Dismas Zelenka, who largely wrote the string parts
                      and added many corrections in org. Moreover, several
                      sections are entirely in Zelenka’s hand:
                      Bars 30–39 in \textit{Kyrie},
                      bars 37–39 in \textit{Gloria},
                      bars~1–3 in \textit{Credo},
                      bars 1–4 in \textit{Sanctus}, and
                      the complete \textit{Benedictus}
                      and \textit{Agnus Dei}. \\
    & –    & ob     & The directives “Tutti” and “Vv.” in the violin parts
                      sometimes indicate the beginning and end of segments
                      where the oboes should play unison with the violins.
                      Based on these directives, the oboe parts of this edition
                      have been assembled. \\
  \midrule
  1 & 5    & vla, A & 4th \quarterNote\ in \B1: e′4 \\
    & 7    & T      & 1st \halfNote\ in \B1: c′2 \\
    & 37   & ob 1   & last \halfNote\ in \B1 unison with vl 1 \\
  \midrule
  2 & 14   & B      & rhythm of 2nd \quarterNote\ in \B1:
                      \eighthNoteDotted–\sixteenthNote \\
    & 16   & clno 1 & 1st \quarterNote\ in \B1: f″4 \\
    & 33   & vl 2   & 6th \eighthNote\ in \B1: f′8 \\
    & 34   & B      & 6th \eighthNote\ in \B1: f8 \\
    & 37ff & –      & Zelenka replaced the originial, shorter \textit{Amen}
                      fugue (5 bars following the fermata in bar 37) by a short
                      \textit{Adagio} section and the longer fugue printed
                      in this edition. \\
    & 37   & timp   & 8th \sixteenthNote\ in \B1: c16 \\
    & 44   & org    & 2nd \quarterNote\ of lower voice in \B1: f′8–e′16–d′16 \\
    & 46   & vl 2   & 1st \halfNote\ in \B1: \halfNoteRest \\
    & 46   & vla, A & 4th \eighthNote\ in \B1: e′8 \\
    & 49   & T      & 14th \sixteenthNote\ in \B1: d′16 \\
  \midrule
  3 & 1–4  & clno, timp & bars missing in \B1 \\
    & 2    & ob 2, vl 2 & 7th \eighthNote\ in \B1: d″8 \\
    & 11   & A      & 1st \halfNote\ in \B1: g′4–g′8–g′8 \\
    & 11   & vla, org & 4th \eighthNote\ in \B1: g8 \\
    & 26   & S      & 4th \quarterNote\ in \B1: \quaverRest–\flat b′8 \\
    & 42   & A      & 1st \halfNote\ in \B1: g′4.–g′8 \\
    & 52   & B      & 1st \quarterNote\ missing in \B1 \\
    & 59   & S      & 2nd \quarterNote\ in \B1: g′8.–g′16 \\
    & 77f  & clno 2 & c″4–e″4–c″2 and e″1 \\
    & 77f  & timp   & bars in \B1: c4–c4–c2 and c1 \\
    & 79–95 & –     & The \textit{Amen} fugue only comprises chorus and org,
                      while the remaining instruments are indicated by
                      the directives “NB Amen si scrive e dal gloria” and
                      “Amen da capo ut in Gloria”. Minor differences
                      to the \textit{Amen} fugue in the \textit{Gloria} occur
                      in S (bar 93, 1st \quarterNote) and org (bar 87,
                      3rd \quarterNote; bar 92, 2nd \halfNote; and bar 93–94).
                      Here, the variants of the latter fuge are reproduced,
                      since only this fugue has been revised by Zelenka. \\
    & 83   & org    & 2nd \quarterNote\ of lower voice in \B1: f′8–e′16–d′16 \\
    & 85   & vl 2   & 1st \halfNote\ in \B1: \halfNoteRest \\
  \midrule
  4 & –    & clno, timp & Voices have been added by the editor
                      in bars 1–4, 18, and 46–73. \\
    & 14   & org    & 3rd \quarterNote\ of lower voice in \B1: f′8–f′8 \\
    & 16   & ob 1, vl 1 & 4th \eighthNote\ in \B1: f″8 \\
    & 16   & T      & 7th \eighthNote\ in \B1: b8 \\
    & 17   & clno 2 & 3rd \eighthNote\ in \B1: \quaverRest \\
    & 62   & B      & 1st \halfNote\ in \B1: \halfNoteRest \\
  \midrule
  5 & –    & clno, timp & Voices have been added by the editor. \\
    & 8    & org    & bar in \B1: G4–\crotchetRest–\halfNoteRest \\
    & 29   & org    & 1st \halfNote\ in \B1: \halfNoteRest \\
    & 43   & org    & last \halfNote\ in \B1: \halfNoteRest \\
    & 47   & ob 1, vl 1, S & 1st to 3rd \halfNote\ in \B1: c″1. \\
    & 47   & ob 2, vl 2, A & 1st to 3rd \halfNote\ in \B1: a′1–g′2 \\
}

\eesToc{}

\eesScore

\end{document}
